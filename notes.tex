\documentclass[11pt,letterpaper]{amsart}

\usepackage{ifpdf,ifxetex,ifluatex}
\ifpdf
    \typeout{^^J *** PDF mode *** } 
    \usepackage[protrusion=true,expansion]{microtype}
    \usepackage[pdftex]{graphicx}
\else
    \typeout{^^J *** DVI mode ***} 
    \usepackage[dvips]{graphicx}
\fi 

\ifluatex
    \typeout{ *** LuaLaTeX *** ^^J}
    \makeatletter
    \makeatother 
    % LuaLaTeX specific code
    \usepackage{unicode-math}
    \usepackage{lualatex-math}
\else\ifxetex
    \typeout{ *** XeLaTeX *** ^^J} 
    % XeLaTeX specific code
\else
    \typeout{ *** LaTeX *** ^^J} 
\fi\fi 

\usepackage{amsmath}
\usepackage{amsthm}
\usepackage{amsfonts}
\usepackage{latexsym,amssymb}
\usepackage{setspace}

\usepackage{fullpage}

%\usepackage[all]{xy}
%\CompileMatrices

\DeclareMathOperator{\Img}{im}
\DeclareMathOperator{\id}{id}
\DeclareMathOperator{\Aut}{Aut}
\DeclareMathOperator{\Aug}{Aug}
\DeclareMathOperator{\Kern}{Kern}
\DeclareMathOperator{\Diff}{Diff}
\DeclareMathOperator{\Symp}{Symp}
\DeclareMathOperator{\Ham}{Ham}
\DeclareMathOperator{\MCG}{Map}
\DeclareMathOperator{\del}{\partial}
\DeclareMathOperator{\delbar}{\overline{\partial}}
\DeclareMathOperator*{\fibersum}{\#}
\DeclareMathOperator*{\union}{\cup}
\newcommand{\Z}{\mathbb{Z}}
\newcommand{\Q}{\mathbb{Q}}
\newcommand{\A}{\mathbb{A}}
\newcommand{\C}{\mathbb{C}}
\newcommand{\R}{\mathbb{R}}
\newcommand{\RP}{\mathbb{R}P}
\newcommand{\CP}{\mathbb{C}P}
\newcommand{\CPbar}{\overline{\mathbb{C}P}}
\newcommand{\D}{D}
\newcommand{\T}{\mathbb{T}}
\newcommand{\Tw}{{\mathbb{T}w}}
\newcommand{\into}{\hookrightarrow}

\newtheorem{theorem}{Theorem}
\newtheorem{corollary}[theorem]{Corollary}
\newtheorem{lemma}[theorem]{Lemma}
\newtheorem{definition}{Definition}
%\newtheorem{lemma*}{Lemma}
\newtheorem{proposition}[theorem]{Proposition}
\newtheorem{conjecture}[theorem]{Conjecture}

\usepackage{hyperref}
\hypersetup{%
  bookmarks=false,%
  % pdftoolbar=false,%
  pdffitwindow=true,%
  pdftitle={CSC 281},%
  pdfauthor={Adam C. Knapp},%
  pdfnewwindow=true,%
  colorlinks=false,%
%  citecolor=black,%
%  filecolor=black,%
%  linkcolor=black,%
%  urlcolor=blue,%
%  pdfhighlight={/I},%
}

\makeatother

\title[CSC-281]%
{Rough and Bare-bones Notes for\\CSC-281 Intro to Computer Science II}

\author[A. C. K.]{Adam C. Knapp}
\address{American University\\ Washington, DC 20016}
\email{\href{mailto:knapp@american.edu}{knapp@american.edu}}

\usepackage{sourcecodepro}
\usepackage{xcolor}
\usepackage{minted}
\usepackage{listings}

\begin{document}
\ifpdf
\DeclareGraphicsExtensions{.mps,.pdf,.jpg,.png}
\else
\DeclareGraphicsExtensions{.mps,.eps,.ps,.jpg,.png}
\fi

\bibliographystyle{plain}

\maketitle

The real documentation is your friend, go there before Googling.
\begin{center}
  \url{https://docs.oracle.com/javase/8/docs/api/}
\end{center}
\section{Java basics}

\begin{minted}[fontsize=\small]{Java}
/*
 A multi-line comment
 another line of comment
 */ 
public class HelloWorld {
  public static void main(String[] args) {
    // a single-line comment
    System.out.println("Hello World!");
  }
}
\end{minted}

Braces and semicolons, like C.

\begin{itemize}
\item Must be in \texttt{HelloWorld.java}
\item Compile with \texttt{javac HelloWorld}
\item Run with \texttt{java HelloWorld}
\end{itemize}

\subsection{Primitive Types}

Variable types are explicity declared:
\begin{itemize}
\item primitive integer types: 8-bit \texttt{byte}, 16-bit
  \texttt{short}, 32-bit \texttt{int}, 64-bit \texttt{long}. All types
  are {\em signed}.
\item primitive character type: \texttt{char} (really same as
  \texttt{int}) character literals look like: \texttt{'w'}
\item primitive floating point types: 32-bit \texttt{float}, 64-bit
  \texttt{double}
\item \texttt{boolean} literals: \texttt{true} and \texttt{false}
\item Strings, string literals look like \texttt{"CSC281"}
\end{itemize}

\subsection{Increment/Decrement}

\begin{itemize}
\item \texttt{a++} increments \texttt{a} by 1 and evaluates to \emph{previous value}
\item \texttt{++a} increments \texttt{a} by 1 and evaluates to \emph{new value}
\item \texttt{a-{}-} decrements \texttt{a} by 1 and evaluates to \emph{previous value}
\item \texttt{-{}-a} decrements \texttt{a} by 1 and evaluates to \emph{new value}
\end{itemize}

The difference between pre- and post-increment can be confusing in
compound statements. Avoid if there is no tangible benefit.

\subsubsection{Pitfalls}

floating point isn't precise
\begin{minted}[fontsize=\small]{Java}
  for(double d = 1.1; d != 2; d += 0.1)
  { System.out.println(d); }
\end{minted}

integers overflow silently 
\begin{minted}[fontsize=\small]{Java}
  for(int i = 1; i < Integer.MAX_VALUE; i *= 2)
  { System.out.println(i); }
\end{minted}

\subsection{Reference Types}

Reference types point at locations in memory. The default value of a
reference type is \texttt{null}. All arrays and objects are reference
types. Example of a built in object: \texttt{String}

Arrays are 0-indexed and declared in the style:
\begin{minted}[fontsize=\small]{Java}
  int[] tenNumbers = new int[10];
\end{minted}
Arrays have elements, referenced using square braces. Ex:
\texttt{tenNumbers} contains \texttt{tenNumbers[0]} through
\texttt{tenNumbers[9]}.


Objects have fields (data) and methods (code) and are {\em non-static
  instances} of classes.

Some useful methods of \texttt{String}'s:
\begin{itemize}
\item \texttt{int length()} Ex: \texttt{"abc".length()} evaluates to 3.
\item \texttt{char charAt(int)} Ex: \texttt{"abc".charAt(1)} evaluates
  to \texttt{'b'}
\item \texttt{String substring(int,int)} Ex:
  \texttt{"abc".substring(0,2)} evaluates to \texttt{"ab"}
\item \texttt{String substring(int)} Ex:
  \texttt{"abc".substring(1)} evaluates to \texttt{"bc"}
\end{itemize}
See
\url{https://docs.oracle.com/javase/8/docs/api/java/lang/String.html}

\subsubsection{Passing reference types}

\begin{minted}[fontsize=\small]{Java}
  void messWithReference(int[] arr) { arr[0] = 1; }
  void dontMessWithValue(int n) { n = 1; }
\end{minted}

The first actually modifies the array passed to it. The second only
changes the value passed to it inside the method, the caller does not
see a change.

\subsection{Booleans and Branching}

boolean valued comparison statements for numeric types \texttt{a>=b},
\texttt{a>b}, \texttt{a<=b}, \texttt{a<b}

comparison for types: \texttt{a==b} (not \texttt{=}) and \texttt{a!=b}

\emph{Note}: for reference types, these compare equality of reference,
{\em not} equality of the contents! For arrays you should instead use
\texttt{Arrays.equals()} or \texttt{Arrays.deepEquals()}. For other
reference types (ex: \texttt{String}'s) use the \texttt{equals()}
method. If you define your own classes, be sure to \emph{implement} an
\texttt{equals()} method.

for boolean types: OR: \texttt{a|b}, short-circuit OR: \texttt{a||b},
AND: \texttt{a\&b}, short-circuit AND: \texttt{a\&\&b}, NOT:
\texttt{!a}

\subsubsection{If statement}
\begin{minted}[fontsize=\small]{Java}
if( test1 ) {
// do something
} else if( test2 ) {
// do something else
} else {
// do yet another, different, thing
}
\end{minted}
The \texttt{else if} and \texttt{else} blocks are optional. Single
line blocks are OK, but may violate some coding styles.


Note: Suppose that \texttt{thingIsTrue()} is a boolean-valued method,
testing something you are interested in. A lot of beginners want to
write
\begin{minted}[fontsize=\small]{Java}
if(thingIsTrue()==true) ...
\end{minted}
This is a bit silly in that you are getting a true/false value and
comparing it to \texttt{true}. Is true true? yes. If false true?
no. You already have your answer. Instead write
\begin{minted}[fontsize=\small]{Java}
if(thingIsTrue()) ...
\end{minted}

\subsubsection{ternary operator} 

\begin{minted}[fontsize=\small]{Java}
booleanExpression ?  valueIfTrue : valueIfFalse
\end{minted}

\subsubsection{switch statement}
\begin{minted}[fontsize=\small]{Java}
switch(n % 2) {
case 0:
  System.out.println("n is even");
  break;
case -1:
  System.out.print("n is negative and ");
case 1:
  System.out.println("n is odd");
  break;
default:
  // unreachable in this case
}
\end{minted}
Jumps to case statement. Note \texttt{break} statements. Execution
cascades through if you don't have them. Question: What if $n=5$, what
if $n=-7$? 

Historical tidbit: \href{https://en.wikipedia.org/wiki/Duff's_device}{Duff's device}

\subsection{Loops}

\begin{itemize}
\item for loop
  \begin{minted}[fontsize=\small]{Java}
    for(initializer; condition; updateStep) {
      // code block
    }
  \end{minted}
  typical:
  \begin{minted}[fontsize=\small]{Java}
    for(int i = 0; i < n; i++) {
      // code block
    }
  \end{minted}
\item for-each variation
  \begin{minted}[fontsize=\small]{Java}
    for(Type item : arrayOrIterableContainingType) {
      // code block
    }
  \end{minted}
\item while loop
  \begin{minted}[fontsize=\small]{Java}
    while( booleanValuedStatement ) {
      // code block
    }
  \end{minted}
\item do-while loop
  \begin{minted}[fontsize=\small]{Java}
    do {
      // code block
    } while( booleanValuedStatement );
  \end{minted}
  note semicolon at end. When you want the block to execute at least
  once. Useful for user input
  
  In any of these loops, you can skip to the next iteration using the
  keyword \texttt{continue}. You can break out of the loop entirely
  using \texttt{break}. Both support breaking out of nested loops /
  continuing an outer loop using labels. e.g.
\begin{minted}[fontsize=\small]{Java}
outerLoop:
while( thing1 ) {
  while( thing2 ) {
    // do something
    if( thing3 ) { break outerLoop; }
  }
}
\end{minted}
  here, the break statement jumps out of both while loops. Try to code
  so that this isn't necessary.
  
\end{itemize}

\section{Fisher-Yates / Knuth Shuffle}

\begin{minted}[fontsize=\small]{Java}
  void shuffle(int[] a) {
    int n = a.length;
    for(int i = 0; i < n - 1; i++) {
      // generate j with i <= j < n
      int j = i + (int)( Math.random()*(n-i) );
      // swap a[i] and a[j]
      int temp = a[i];
      a[i] = a[j];
      a[j] = temp;
    }
  }
\end{minted}


\end{document}

%%% Local Variables: 
%%% mode: latex
%%% TeX-master: t
%%% LaTeX-command: "latex -shell-escape"
%%% End: 
